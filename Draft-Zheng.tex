\documentclass{article}
\usepackage{fullpage}
\usepackage{hyperref}
\usepackage{indentfirst}
\usepackage[utf8]{inputenc}

\title{Computational reproducibility in scientific discovery: extending PRIMAD through vocabulary and workflows}
\author{GaoZheng Liu\footnote{Corresponding author (\texttt{gl11@illinois.edu}).}, Santiago N\'u\~nez-Corrales and Bertram Lud\"ascher}
\date{July 2019}

\begin{document}

\maketitle
\begin{abstract}
    Lalala
\end{abstract}

\section{Introduction}

\section{Reproducibility in the context of computational-driven discovery}

\subsection{Reproducibility and the publishing pipeline}

\subsection{Reproducibility and the scientific pipeline}

\section{What do we mean by reproducibility?}

\subsection{Defining reproducibility}

\subsection{Models of reproducibility}

\subsection{PRIMAD}

\subsection{Other models of reproducibility}

\section{Applying reproducibility}

\subsection{Journal badging and artifact execution virtual platforms}

\subsection{Domain-centered multiple reproducibility studies}

\section{Extending PRIMAD: workflows}

\section{PRIMAD-core: a controlled vocabulary for reproducibility studies}

\subsection{Objectives}

\subsection{Vocabulary}

\subsection{Automation}

\subsection{Example 1}

\subsection{Example 2}

\section{PRIMAD-core as a facilitator for reproducibility in the Whole Tale}
read: Understand others work, get more citation (for authors), build upon others research more easily.Trust.
write: develop research that is more likely to be understood by others with less effort.Trust

\section{Conclusions and next steps}
Other platform 

\section{Acknowledgments}

\bibliographystyle{unsrt}
\bibliography{references}

\end{document}
