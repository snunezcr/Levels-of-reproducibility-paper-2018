\documentclass{article}
\usepackage{fullpage}
\usepackage{hyperref}
\usepackage{indentfirst}
\usepackage[utf8]{inputenc}

\title{Computational reproducibility in scientific discovery: extending PRIMAD through vocabulary and workflows}
\author{GaoZheng Liu\footnote{Corresponding author (\texttt{gl11@illinois.edu}).}, Santiago N\'u\~nez-Corrales and Bertram Lud\"ascher}
\date{July 2019}

\begin{document}
-------------------------------------------- cut
\section{Introduction}

\section{Reproducibility in the context of computational-driven discovery}
high level concept
\subsection{Defining reproducibility}
\subsection{Benefits/function of reproducibility}
Reproducibility and the scientific pipeline
Reproducibility and the publishing pipeline
\subsection{Hypothesis and reproducibility result verification}
\subsection{Domain-centered multiple reproducibility studies}
\subsection{Journal badging} 
\subsection{artifact execution virtual platforms}

\section{Models of reproducibility}
\subsection{PRIMAD}
\subsection{Other models of reproducibility}
\subsection{extended PRIMAD model}
\section{What do we mean by reproducibility?}

\section{Applying reproducibility}


\section{Extending PRIMAD: workflows}

\section{PRIMAD-core: a controlled vocabulary for reproducibility studies}

\subsection{Objectives}

\subsection{Vocabulary}

\subsection{Automation}

\subsection{Example 1}

\subsection{Example 2}

\section{PRIMAD-core as a facilitator for reproducibility in the Whole Tale}
read: Understand others work, get more citation (for authors), build upon others research more easily.Trust.
write: develop research that is more likely to be understood by others with less effort.Trust

\section{Conclusions and next steps}
Other platform 

\section{Acknowledgments}

\bibliographystyle{unsrt}
\bibliography{references}

\end{document}
